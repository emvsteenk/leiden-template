%%% here we rewrite everything into proper text %%%

The Crimson Petal and the White is a historical fiction novel by Michel Faber,
which was written in 2002, but set in 1875 \autocite{Guardian}. The title refers
to the first line of a poem by Alfred, Lord Tennyson, but Faber says the poem
had no further influence on the novel, saying that he ``just liked the symbolism
of crimson and white petals -- blood, sexual immorality, purity, snow, the
perfume business, and so on''\autocite{3am}. The novel follows a young man,
William Rackham, who has to transition away from his literary dreams into the
world of the perfume business, who meets a prostitute called Sugar, who is
rumoured to make anybody's wildest dreams come true. Rackham buys Sugar, and
installs her in his household as the governess to his daughter, all the while
his unknowing (child-bride) wife suffers from psychotic episodes due to her lack
of sexual education.

While the novel is set in the Victorian era, it's narrator is modern, and acts
as a dominant, all knowing guide that leads you, the unnowing alien, around
their home. The novel uses an \textit{authorial narrative
situation}~\autocite{basic} where the narrator is not a character in the story,
but someone who tells the story directly to you (\textit{ad
spectatores}~\autocite{basic}) in the \textit{gnomic present}~\autocite{basic},
where they know all the facts, and everything you, the reader, think or know is
wrong.

%%% Source text analysis %%%

This translation focuses on the first 35 lines of the novel, in which no
character is directly named, and most of the text revolves around the narrator,
\textsc{i}, and the reader, \textsc{you}. I will therefore use \textsc{small
capitals} to indicate when I am talking about these characters in the third
person. I will use the pronoun they for both, as their genders are not made
explicit in the text and the gender of the reader is unknown to the writer.
Assumptions could be made, but it seems to me that Faber made an explicit choice
for the gender and sexual orientation of \textsc{i} and \textsc{you} to remain
ambiguous, so as not to exclude any of his readers from experiencing the effects
of this text properly. A lot of work is put into this being an immersive
experience for the reader. \textsc{i} is aware of the existence of the reader,
whom they address directly as \textit{you}, which places \textsc{You}, or, the
reader, directly into the story.

Two things about the text are striking: first, the narrator is aware of the
reader, and of the fact that this is a story. Second, the narrator insists that
everything you think you know about the Victorian era is wrong, and they know
better, thus expressing and excerting power over you. Both of these elements are
used by the narrator to unsettle the reader, making the reader feel insecure,
like they are stumbling blindly around Victorian era London, knowing nothing and
being thrown off by everything they see.

The unsettling feeling, making the reader feel like they know nothing, and that
they are being dragged left and right around the story, having no choice, is
expressed through all four categories described by ~\textcite{leechshort}:
lexical features, grammatical features, schemes and tropes, and context and
cohesion.

%%% Lexical features %%%







and is being told what sensory experiences they are having.
